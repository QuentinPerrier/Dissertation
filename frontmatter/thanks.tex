% the acknowledgments section
%1) Ecrire les remerciements
%2) Demander à Philippe 


%Même si une thèse reste un travail très solitaire, elle ne peut s'accomplir dans une complète autarcie. 
Cette page est dédiée à toutes les personnes m'ayant permis directement ou indirectement de mener à bout ce projet de thèse.
Me voici donc lancé dans la fameuse série de remerciements, la partie de la thèse écrite en dernier par l'auteur et souvent celle qui est lue en premier par les lecteurs (quand elle n'en constitue pas la seule).  
 
Je tiens à remercier en premier lieu Philippe Quirion, mon directeur de thèse, sans qui aucun des travaux présentés ici n'auraient vu le jour.
L'aboutissement de cette thèse tient en grande partie à sa confiance, à sa capacité à me lancer sur de nouvelles pistes toujours prometteuses, et à m'accompagner ensuite pour qu'elles aboutissent.  
Son expertise et son esprit critique n'ont d'égal que son humilité, fait particulièrement appréciable dans le monde académique. 
A ces qualités s'ajoutent la gentillesse, la décontraction et la disponibilité, qui en font un directeur de thèse exceptionnel.

Je remercie ensuite chaleureusement Matthieu Glachant (CERNA) et Joachim Scheich (Grenoble Ecole de Management) d'avoir accepté d'être rapporteurs de ma thèse, ainsi que les autres membres du jury Stéphanie Monjon (Dauphine), Roger Guesnerie (Collège de France) et Thomas Sterner (University of Gothenburg).

J'exprime également ma gratitude à la Smash, pour m'avoir grandement facilité la possibilité de partir en conférence à l'étranger (ce dont j'ai bien profité), la Commission Européenne pour le financement du projet Cecilia 2050, et bien sûr Agroparistech et le ministère de l'Agriculture pour le financement de ma thèse. 

Je tiens aussi à remercier mes co-auteurs: Philippe, Gaëtan, Céline, Oskar et Marie-Laure du CIRED, Onno Kuik (VU University Amsterdam ), Oliver Sartor (IDDRI), Jean-Pierre Ponssard (Ecole Polytechnique), Julien Chevallier (IPAG) et enfin Misato Sato (LSE), que je remercie tout particulièrement avec Antoine Dechezleprêtre pour m'avoir accueilli au Grantham Institute à Londres pour un visiting de quatre mois.

Mes pensées vont ensuite à la grande famille du CIRED: son incarnation Jean-Charles Hourcade et son  directeur Franck Lecocq pour m'y avoir accueilli (et également pour ce dernier son aide lors de l'oral de la Commission de la Formation Doctorale m'ayant permis d'être accepté en thèse); Eleonore, Naceur et Yaël qui font tourner la boutique; Céline et Gaëtan qui m'ont ouvert la porte de ce laboratoire lors de mon stage de master. 
Je pense aussi aux nombreux autres collègues et amis: Aurélien, Cédric (à qui je dois mon vélo ayant remplacé mon pass Navigo), Manu, Laurent, Vincent, les américains Julie, Adrien et Paolo, mais aussi Marie-Laure, Baptiste, Fabrice, Manon, William, Gaëlle, Florent, Jules, Meriem et bien d'autres.
Le CIRED est véritablement un endroit où il fait bon vivre et travailler, et où les conversations à midi ou à la machine à café n'ont aucune limite. 


Mes pensées vont ensuite à mes proches. 
Beaucoup se demandaient ce que je faisais de mes journées (``concrètement au travail tu fais \emph{quoi}?''), et m'ont cru en semi-vacances pendant trois ans ou en "thèse youtube", avec un fond de vérité. 
Parmi les marseillais, un immense merci à Agnès et JB, à tous les "gros", et à Jean-Marie.
Parmi les parisiens, je pense particulièrement aux coincheurs de l'extrême Simon BB, Xavier et Raph, mais aussi à Arthur, Michalle, Lily, Elodie, David, Célian et Youssef.
J'ai aussi une pensée pour mes collègues du Corps et amis Ghislain, Fred C et Loïc; et pour mes anciens et actuels colocs Antoine, Guillaume, Maisie (que je remercie aussi pour les quelques boulots d'édition), Anne-Claire et Pierrick, dont les fins de thèse approchent à grands pas.
Je pense aussi à Simon F. et lui souhaite le meilleur rétablissement possible.  
 
  
Avant de passer à la famille, je remercie Pierre et Sylviane, pour leur gentillesse et leur générosité.
Je remercie mes parents pour leur soutien indéfectible et mes frères et soeurs, Nicolas (auquel je souhaite tous mes voeux de bonheur avec Marie-Sophie), Damien et Marine.  
J'ai également une pensée pour mes cousins François (heureusement qu'il m'a montré le Scrabble en fin de thèse et pas au début) et Alex (dont le périple me faisait envie pendant la rédaction de ce manuscrit), et mes grand-parents Liliane, Catherine et Jacques.

Enfin, merci infiniment à Jamora pour son amour et pour être à mes côtés. 
  